% Use the following line _only_ if you're still using LaTeX 2.09.
%\documentstyle[icml2011,epsf,natbib]{article}
% If you rely on Latex2e packages, like most moden people use this:
\documentclass{article}

% For figures
\usepackage{graphicx} % more modern
%\usepackage{epsfig} % less modern
\usepackage{subfigure} 

% For citations
\usepackage{natbib}

% For algorithms
\usepackage{algorithm}
\usepackage{algorithmic}

% As of 2010, we use the hyperref package to produce hyperlinks in the
% resulting PDF.  If this breaks your system, please commend out the
% following usepackage line and replace \usepackage{icml2011} with
% \usepackage[nohyperref]{icml2011} above.
\usepackage{hyperref}

% Packages hyperref and algorithmic misbehave sometimes.  We can fix
% this with the following command.
\newcommand{\theHalgorithm}{\arabic{algorithm}}

% Employ the following version of the ``usepackage'' statement for
% submitting the draft version of the paper for review.  This will set
% the note in the first column to ``Under review.  Do not distribute.''
\usepackage[accepted]{icml2011} 
% Employ this version of the ``usepackage'' statement after the paper has
% been accepted, when creating the final version.  This will set the
% note in the first column to ``Appearing in''
% \usepackage[accepted]{icml2011}


% The \icmltitle you define below is probably too long as a header.
% Therefore, a short form for the running title is supplied here:
\icmltitlerunning{Context Aware Movie Recommender Systems}

\begin{document} 

\twocolumn[
\icmltitle{CAMR - Context Aware Movie Recommender Systems}

% It is OKAY to include author information, even for blind
% submissions: the style file will automatically remove it for you
% unless you've provided the [accepted] option to the icml2011
% package.
\icmlauthor{Primal Pappachan}{primal1@umbc.edu}
 %\icmladdress{Your Fantastic Institute,
%            314159 Pi St., Palo Alto, CA 94306 USA}
\icmlauthor{Arnav Joshi}{arnavj1@umbc.edu}
%\icmladdress{Their Fantastic Institute,
%            27182 Exp St., Toronto, ON M6H 2T1 CANADA}

% You may provide any keywords that you 
% find helpful for describing your paper; these are used to populate 
% the "keywords" metadata in the PDF but will not be shown in the document
\icmlkeywords{recommender systems, machine learning}

\vskip 0.3in
]

\section{Introduction}
The task of recommender systems is to turn data on users and their preferences into predictions of possible future likes and interests \cite{adomavicius2011context}.  The majority of existing approaches to recommender systems focuses on recommending the most relevant items to individual users. However, they do not take into consideration any contextual information, such as time, place and the company of other people (in use cases such as movies). In other words, traditional recommender systems deal with applications having only two types of entities, users and items, and do not put them into a context when providing recommendations. 
Context information is any information about the situation, circumstances and user state when a user is consuming the content item. Context can be the time of day, weather, social situation, user’s mood etc.\cite{segaran2008programming} However, the relationship between context and user decision making is very complex and difficult to model. 
Context-aware recommender systems help users and their desired content in a reasonable time, by exploiting the pieces of information that describe the situation in which users will consume the items. CAMRS models this approach by providing the recommendation on the basis of a function of user, item, previous ratings, as well as the context information provided of the user.

\subsection{Importance of context} 
The inclusion of the contextual information into the recommendation process presents opportunities for richer and more diverse interactions between the end-users and recommender systems. With the help of user's context, CAMRS will not only incorporate the user's profile of previous movies he has watched, but also include domain-dependent context modeling, which will help give better recommendation on the selection of movies.  Adapting the item choice based on the user's context can help give us better ratings for each movie recommendation.

\subsection{Collaborative Filtering}
Collaborative filtering systems gather item ratings as a form of user feedback for items in a given domain and exploit similarities and differences among profiles of several users in determining how to recommend an item. In movie recommendation, users provide ratings for the movies they have watched. 
A rating function in a recommendation system is one which tries to estimate the rating for the new item (here, the movie choice) based on the user’s profile i.e. previous preferences.
This is done by initially generating a two-dimensional User x Item recommender matrix which takes partial user preference data as its input and produces a list of recommendations for each user as an output.
After the recommendation function is defined (or constructed) based on the available data, recommendation list for any given user u is typically generated by using the recommendation function on user u and all candidate items to obtain a predicted rating for each of the items and then by ranking all items according to their predicted rating value.
In CAMRS we hope to gain additional insight on user preferences by taking contextual information into consideration as explicit categories of data, such as the time, location and social situation(with a companion or not). The rating function r can thus be defined as: 
\begin{equation}r: User x Item x Context = Rating 
\end{equation}\cite{adomavicius2011context} 
The recommendation will be on a Likert scale (scale of 1-5) and detection of movie relevance is done on the basis of rating information. 

\subsection{Content-based Filtering}
Content based filtering methods provide recommendations by comparing representations of content contained in an item to representations of content that interests the user.

\subsection{Post Filtering based on contextual attributes}
CAMRS generates recommendations based on contextual post-filtering. Post-filtering is a type of neighborhood-based algorithm, where a subset of users is chosen based on their similarity to the active user (the context is initially ignored). The resulting set of recommendations is adjusted (contextualized) for each user using the contextual information. The algorithm can be summarized in the following steps:
\begin{enumerate}
 \item Similarity between users is measured as the Pearson correlation between their ratings vectors.
 \item Select n users that have the highest similarity with the active user.
 \item Compute a prediction from a weighted combination of the selected neighbors’ ratings.
\end{enumerate}

\section{Related Work}
Melville et al. [7] incorporated both content based and collaborative filtering methods for recommender systems. Their approach uses a content-based predictor to enhance existing user data, and then provide personalized suggestions through collaborative filtering.

\section{Proposed method}
\subsection{Dataset}
We plan to use the LDOS - CoMoDa [1] dataset, made available by Dr. Andrej Kosir from the University of Ljubljana, Slovenia. The LDOS - CoMoDa dataset is a context-rich movie recommender dataset, which contains ratings for movies along with contextual information describing the situations in which the movies were watched. 
It contains 30 variables among which are 12 contextual variables. Other variables are general user information (age, sex, city, country) and content (movie) metadata (director, movieCountry, movieLanguage, movieYear, genre1, genre2, genre3, actor1, actor2, actor3, budget).


\section{Experiments}

\section{Conclusions}

\bibliography{pr_ref}
\bibliographystyle{icml2011}

\end{document} 
